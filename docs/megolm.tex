\documentclass[10pt]{article}
\usepackage{hyperref}
\usepackage{amsmath}
\usepackage{tabularx}


\newcommand\tmpset[2]{
  \begingroup
  #1
  #2
  \endgroup
}

\newcommand\autoscaletable[3][0pt]{
  \tmpset{\setlength\tabcolsep{#1}}{
    \noindent
    \begin{tabular*}{\textwidth}{@{\extracolsep{\fill}} #2}
      #3
    \end{tabular*}}
}

\begin{document}

\section{Megolm group ratchet}\label{megolm-group-ratchet}

An AES-based cryptographic ratchet intended for group communications.

\subsection{Background}\label{background}

The Megolm ratchet is intended for encrypted messaging applications
where there may be a large number of recipients of each message, thus
precluding the use of peer-to-peer encryption systems such as
\href{https://gitlab.matrix.org/matrix-org/olm/blob/master/docs/olm.md}{Olm}.

It also allows a recipient to decrypt received messages multiple times.
For instance, in client/server applications, a copy of the ciphertext
can be stored on the (untrusted) server, while the client need only
store the session keys.

\subsection{Overview}\label{overview}

Each participant in a conversation uses their own outbound session for
encrypting messages. A session consists of a ratchet and an
\href{http://ed25519.cr.yp.to/}{Ed25519} keypair.

Secrecy is provided by the ratchet, which can be wound forwards but not
backwards, and is used to derive a distinct message key for each
message.

Authenticity is provided via Ed25519 signatures.

The value of the ratchet, and the public part of the Ed25519 key, are
shared with other participants in the conversation via secure
peer-to-peer channels. Provided that peer-to-peer channel provides
authenticity of the messages to the participants and deniability of the
messages to third parties, the Megolm session will inherit those
properties.

\subsection{The Megolm ratchet
algorithm}\label{the-megolm-ratchet-algorithm}

The Megolm ratchet \(R_i\) consists of four parts, \(R_{i,j}\) for
\(j \in {0,1,2,3}\). The length of each part depends on the hash
function in use (256 bits for this version of Megolm).

The ratchet is initialised with cryptographically-secure random data,
and advanced as follows:

$\begin{aligned}
R_{i,0} &=
  \begin{cases}
  H_0\left(R_{2^{24}(n-1),0}\right) &\text{if }\exists n | i = 2^{24}n\\
  R_{i-1,0} &\text{otherwise}
  \end{cases}\\
R_{i,1} &=
  \begin{cases}
  H_1\left(R_{2^{24}(n-1),0}\right) &\text{if }\exists n | i = 2^{24}n\\
  H_1\left(R_{2^{16}(m-1),1}\right) &\text{if }\exists m | i = 2^{16}m\\
  R_{i-1,1} &\text{otherwise}
  \end{cases}\\
R_{i,2} &=
  \begin{cases}
  H_2\left(R_{2^{24}(n-1),0}\right) &\text{if }\exists n | i = 2^{24}n\\
  H_2\left(R_{2^{16}(m-1),1}\right) &\text{if }\exists m | i = 2^{16}m\\
  H_2\left(R_{2^8(p-1),2}\right) &\text{if }\exists p | i = 2^8p\\
  R_{i-1,2} &\text{otherwise}
  \end{cases}\\
R_{i,3} &=
  \begin{cases}
  H_3\left(R_{2^{24}(n-1),0}\right) &\text{if }\exists n | i = 2^{24}n\\
  H_3\left(R_{2^{16}(m-1),1}\right) &\text{if }\exists m | i = 2^{16}m\\
  H_3\left(R_{2^8(p-1),2}\right) &\text{if }\exists p | i = 2^8p\\
  H_3\left(R_{i-1,3}\right) &\text{otherwise}
  \end{cases}
\end{aligned}$

where \(H_0\), \(H_1\), \(H_2\), and \(H_3\) are different hash
functions. In summary: every \(2^8\) iterations, \(R_{i,3}\) is reseeded
from \(R_{i,2}\). Every \(2^{16}\) iterations, \(R_{i,2}\) and
\(R_{i,3}\) are reseeded from \(R_{i,1}\). Every \(2^{24}\)
iterations, \(R_{i,1}\), \(R_{i,2}\) and \(R_{i,3}\) are reseeded
from \(R_{i,0}\).

The complete ratchet value, \(R_{i}\), is hashed to generate the keys
used to encrypt each message. This scheme allows the ratchet to be
advanced an arbitrary amount forwards while needing at most 1020 hash
computations. A client can decrypt chat history onwards from the
earliest value of the ratchet it is aware of, but cannot decrypt history
from before that point without reversing the hash function.

This allows a participant to share its ability to decrypt chat history
with another from a point in the conversation onwards by giving a copy
of the ratchet at that point in the conversation.

\subsection{The Megolm protocol}\label{the-megolm-protocol}

\subsubsection{Session setup}\label{session-setup}

Each participant in a conversation generates their own Megolm session. A
session consists of three parts:

\begin{itemize}
\item
  a 32 bit counter, \(i\).
\item
  an \href{http://ed25519.cr.yp.to/}{Ed25519} keypair, \(K\).
\item
  a ratchet, \(R_i\), which consists of four 256-bit values, \(R_{i,j}\)
  for \(j \in {0,1,2,3}\).
\end{itemize}

The counter \(i\) is initialised to \(0\). A new Ed25519 keypair is
generated for \(K\). The ratchet is simply initialised with 1024 bits of
cryptographically-secure random data.

A single participant may use multiple sessions over the lifetime of a
conversation. The public part of \(K\) is used as an identifier to
discriminate between sessions.

\subsubsection{Sharing session data}\label{sharing-session-data}

To allow other participants in the conversation to decrypt messages, the
session data is formatted as described in
\ref{session-sharing-format}{Session-sharing format}. It
is then shared with other participants in the conversation via a secure
peer-to-peer channel (such as that provided by
\href{https://gitlab.matrix.org/matrix-org/olm/blob/master/docs/olm.md}{Olm}).

When the session data is received from other participants, the recipient
first checks that the signature matches the public key. They then store
their own copy of the counter, ratchet, and public key.

\subsubsection{Message encryption}\label{message-encryption}

This version of Megolm uses
\href{http://csrc.nist.gov/publications/fips/fips197/fips-197.pdf}{AES-256}
in
\href{http://csrc.nist.gov/publications/nistpubs/800-38a/sp800-38a.pdf}{CBC}
mode with \href{https://tools.ietf.org/html/rfc2315}{PKCS\#7} padding
and \href{https://tools.ietf.org/html/rfc2104}{HMAC-SHA-256} (truncated
to 64 bits). The 256 bit AES key, 256 bit HMAC key, and 128 bit AES IV
are derived from the megolm ratchet \(R_i\):

$\begin{aligned}
 \mathit{AES\_KEY}_{i}\;\parallel\;\mathit{HMAC\_KEY}_{i}\;\parallel\;\mathit{AES\_IV}_{i}
    &= \operatorname{HKDF}\left(0,\,R_{i},\text{"MEGOLM\_KEYS"},\,80\right) \\
\end{aligned}$

where \(\parallel\) represents string splitting, and
\(\operatorname{HKDF}\left(\mathit{salt},\,\mathit{IKM},\,\mathit{info},\,L\right)\)
refers to the \href{https://tools.ietf.org/html/rfc5869}{HMAC-based key
derivation function} using using
\href{https://tools.ietf.org/html/rfc6234}{SHA-256} as the hash function
(\href{https://tools.ietf.org/html/rfc5869}{HKDF-SHA-256}) with a salt
value of \(\mathit{salt}\), input key material of \(\mathit{IKM}\),
context string \(\mathit{info}\), and output keying material length of
\(L\) bytes.

The plain-text is encrypted with AES-256, using the key
\(\mathit{AES\_KEY}_{i}\) and the IV \(\mathit{AES\_IV}_{i}\) to give
the cipher-text, \(X_{i}\).

The ratchet index \(i\), and the cipher-text \(X_{i}\), are then packed
into a message as described in
\ref{message-format}{Message format}. Then the entire
message (including the version bytes and all payload bytes) are passed
through HMAC-SHA-256. The first 8 bytes of the MAC are appended to the
message.

Finally, the authenticated message is signed using the Ed25519 keypair;
the 64 byte signature is appended to the message.

The complete signed message, together with the public part of \(K\)
(acting as a session identifier), can then be sent over an insecure
channel. The message can then be authenticated and decrypted only by
recipients who have received the session data.

\subsubsection{Advancing the ratchet}\label{advancing-the-ratchet}

After each message is encrypted, the ratchet is advanced. This is done
as described in \ref{the-megolm-ratchet-algorithm}{The
Megolm ratchet algorithm}, using the following definitions:

$\begin{aligned}
    H_0(A) &\equiv \operatorname{HMAC}(A,\text{``\char`\\x00"}) \\
    H_1(A) &\equiv \operatorname{HMAC}(A,\text{``\char`\\x01"}) \\
    H_2(A) &\equiv \operatorname{HMAC}(A,\text{``\char`\\x02"}) \\
    H_3(A) &\equiv \operatorname{HMAC}(A,\text{``\char`\\x03"}) \\
\end{aligned}$

where \(\operatorname{HMAC}(A, T)\) is the HMAC-SHA-256 of \texttt{T},
using \texttt{A} as the key.

For outbound sessions, the updated ratchet and counter are stored in the
session.

In order to maintain the ability to decrypt conversation history,
inbound sessions should store a copy of their earliest known ratchet
value (unless they explicitly want to drop the ability to decrypt that
history - see \ref{partial-forward-secrecy}{Partial
Forward Secrecy}). They may also choose to cache calculated ratchet
values, but the decision of which ratchet states to cache is left to the
application.

\subsection{Data exchange formats}\label{data-exchange-formats}

\subsubsection{Session-sharing format}\label{session-sharing-format}

The Megolm key-sharing format is as follows:

\begin{verbatim}
+---+----+--------+--------+--------+--------+------+-----------+
| V | i  | R(i,0) | R(i,1) | R(i,2) | R(i,3) | Kpub | Signature |
+---+----+--------+--------+--------+--------+------+-----------+
0   1    5        37       69      101      133    165         229   bytes
\end{verbatim}

The version byte, \texttt{V}, is \texttt{"\textbackslash{}x02"}.

This is followed by the ratchet index, \(i\), which is encoded as a
big-endian 32-bit integer; the ratchet values \(R_{i,j}\); and the
public part of the Ed25519 keypair \(K\).

The data is then signed using the Ed25519 keypair, and the 64-byte
signature is appended.

\subsubsection{Message format}\label{message-format}

Megolm messages consist of a one byte version, followed by a variable
length payload, a fixed length message authentication code, and a fixed
length signature.

\begin{verbatim}
+---+------------------------------------+-----------+------------------+
| V | Payload Bytes                      | MAC Bytes | Signature Bytes  |
+---+------------------------------------+-----------+------------------+
0   1                                    N          N+8                N+72   bytes
\end{verbatim}

The version byte, \texttt{V}, is \texttt{"\textbackslash{}x03"}.

The payload uses a format based on the
\href{https://developers.google.com/protocol-buffers/docs/encoding}{Protocol
Buffers encoding}. It consists of the following key-value pairs:

\autoscaletable{c c c c}{\textbf{Name} & \textbf{Tag} & \textbf{Type} & \textbf{Meaning} \\ Message-Index & 0x08 & Integer & The index of the ratchet, $i$ \\
Cipher-Text & 0x12 & String & The cipher-text, $X_i$, of the message \\}

Within the payload, integers are encoded using a variable length
encoding. Each integer is encoded as a sequence of bytes with the high
bit set followed by a byte with the high bit clear. The seven low bits
of each byte store the bits of the integer. The least significant bits
are stored in the first byte.

Strings are encoded as a variable-length integer followed by the string
itself.

Each key-value pair is encoded as a variable-length integer giving the
tag, followed by a string or variable-length integer giving the value.

The payload is followed by the MAC. The length of the MAC is determined
by the authenticated encryption algorithm being used (8 bytes in this
version of the protocol). The MAC protects all of the bytes preceding
the MAC.

The length of the signature is determined by the signing algorithm being
used (64 bytes in this version of the protocol). The signature covers
all of the bytes preceding the signature.

\subsection{Limitations}\label{limitations}

\subsubsection{Message Replays}\label{message-replays}

A message can be decrypted successfully multiple times. This means that
an attacker can re-send a copy of an old message, and the recipient will
treat it as a new message.

To mitigate this it is recommended that applications track the ratchet
indices they have received and that they reject messages with a ratchet
index that they have already decrypted.

\subsubsection{Lack of Transcript
Consistency}\label{lack-of-transcript-consistency}

In a group conversation, there is no guarantee that all recipients have
received the same messages. For example, if Alice is in a conversation
with Bob and Charlie, she could send different messages to Bob and
Charlie, or could send some messages to Bob but not Charlie, or vice
versa.

Solving this is, in general, a hard problem, particularly in a protocol
which does not guarantee in-order message delivery. For now it remains
the subject of future research.

\subsubsection{Lack of Backward
Secrecy}\label{lack-of-backward-secrecy}

\href{https://intensecrypto.org/public/lec_08_hash_functions_part2.html\#sec-forward-and-backward-secrecy}{Backward
secrecy} (also called `future secrecy' or `post-compromise security') is
the property that if current private keys are compromised, an attacker
cannot decrypt future messages in a given session. In other words, when
looking \textbf{backwards} in time at a compromise which has already
happened, \textbf{current} messages are still secret.

By itself, Megolm does not possess this property: once the key to a
Megolm session is compromised, the attacker can decrypt any message that
was encrypted using a key derived from the compromised or subsequent
ratchet values.

In order to mitigate this, the application should ensure that Megolm
sessions are not used indefinitely. Instead it should periodically start
a new session, with new keys shared over a secure channel.

\subsubsection{Partial Forward Secrecy}\label{partial-forward-secrecy}

\href{https://intensecrypto.org/public/lec_08_hash_functions_part2.html\#sec-forward-and-backward-secrecy}{Forward
secrecy} (also called `perfect forward secrecy') is the property that if
the current private keys are compromised, an attacker cannot decrypt
\emph{past} messages in a given session. In other words, when looking
\textbf{forwards} in time towards a potential future compromise,
\textbf{current} messages will be secret.

In Megolm, each recipient maintains a record of the ratchet value which
allows them to decrypt any messages sent in the session after the
corresponding point in the conversation. If this value is compromised,
an attacker can similarly decrypt past messages which were encrypted by
a key derived from the compromised or subsequent ratchet values. This
gives `partial' forward secrecy.

To mitigate this issue, the application should offer the user the option
to discard historical conversations, by winding forward any stored
ratchet values, or discarding sessions altogether.

\subsubsection{Dependency on secure channel for key
exchange}\label{dependency-on-secure-channel-for-key-exchange}

The design of the Megolm ratchet relies on the availability of a secure
peer-to-peer channel for the exchange of session keys. Any
vulnerabilities in the underlying channel are likely to be amplified
when applied to Megolm session setup.

For example, if the peer-to-peer channel is vulnerable to an unknown
key-share attack, the entire Megolm session become similarly vulnerable.
For example: Alice starts a group chat with Eve, and shares the session
keys with Eve. Eve uses the unknown key-share attack to forward the
session keys to Bob, who believes Alice is starting the session with
him. Eve then forwards messages from the Megolm session to Bob, who
again believes they are coming from Alice. Provided the peer-to-peer
channel is not vulnerable to this attack, Bob will realise that the
key-sharing message was forwarded by Eve, and can treat the Megolm
session as a forgery.

A second example: if the peer-to-peer channel is vulnerable to a replay
attack, this can be extended to entire Megolm sessions.

\subsection{License}\label{license}

The Megolm specification (this document) is licensed under the Apache
License, Version 2.0 http://www.apache.org/licenses/LICENSE-2.0.

\end{document}
